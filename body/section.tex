\section{公式}

数学论文必然涉及不少数学公式的使用。下面简单介绍一个可能用得上的数学环境。

首先是行内公式,例如 $ \theta $ 是角度。行内公式使用 \verb|$  $| 包裹。

行间公式不需要编号的可以使用 \verb|\[  \]| 包裹,例如
\[
E=mc^2
\]

其中 $ E $ 是能量,$ m $ 是质量,$ c $ 是光速。

如果希望某个公式带编号,并且在后文中引用可以参考下面的写法:
\begin{equation}
E=mc^2
\label{eq:energy}
\end{equation}

式\cref{eq:energy}是质能方程。

多行公式有时候希望能够在特定的位置对齐,以下是其中一种处理方法。
\begin{align}
P & = UI \\
& = I^2R
\end{align}

\verb|&| 是对齐的位置, \verb|&| 可以有多个,但是每行的个数要相同。

矩阵的输入也不难。
\[
\mathbf{X} = \left(
    \begin{array}{cccc}
    x_{11} & x_{12} & \ldots & x_{1n}\\
    x_{21} & x_{22} & \ldots & x_{2n}\\
    \vdots & \vdots & \ddots & \vdots\\
    x_{n1} & x_{n2} & \ldots & x_{nn}\\
    \end{array} \right)
\]

分段函数这些可以用 \verb|case| 环境,但是它要放在数学环境里面。
\[
f(x) =
    \begin{cases}
        0 &  x \text{为无理数} ,\\
        1 &  x \text{为有理数} .
    \end{cases}
\]

在数学环境里面,字体用的是数学字体,一般与正文字体不同。假如要公式里面有个别文字,则需要把这部分放在 \verb|text| 环境里面,即 \verb|\text{文本环境}| 。

公式中个别需要加粗的字母可以用 \verb|$\bm{math symbol}$| 。如 $ \alpha a\bm{\alpha a} $ 。

以上仅简单介绍了基础的使用,对于更复杂的需求,可以阅读相关的宏包手册,如 \href{http://texdoc.net/texmf-dist/doc/latex/amsmath/amsldoc.pdf}{amsmath}。复杂的公式也不难打出。

\begin{equation}\begin{aligned}
P(y\mid X,\beta,\sigma^2)P(\beta)P(\sigma^2) & =\prod_{i=1}^n\frac{1}{\sqrt{2\pi\sigma^2}}\exp\left(-\frac{(y_i-x_i\beta)^2}{2\sigma^2}\right)\times\mathcal{N}(\beta\mid0,\tau^2) \\
 & \times\text{Inverse-Gamma}(\sigma^2\mid a,b)
\end{aligned}\end{equation}

% \begin{equation}
\begin{align}
    \mathbf{r}_\odot(t)  = & \mathbf{r}_0  + \mathbf{v}_0(t-t_0) + \frac{1}{2}\mathbf{a}_0(t-t_0)^2 + \frac{1}{6}\mathbf{j}_0(t-t_0)^3 \\
      &  + \int_{t_0}^t\int_{t_0}^\tau\left[\sum_{k=1}^4\mathbf{a}_k(\tau') + \mathcal{O}(c^{-4})\right]d\tau'd\tau \quad \text{(日心惯性系)} \\
      \mathbf{r}_\oplus(t) = & \left[\mathbf{r}_\odot(t) - \mathbf{R}_\odot^\oplus(t)\right] \cdot \mathbf{R}_{prec}(t) + \Delta\mathbf{r}_{nut}(t)\quad \text{(地心惯性系)}
\end{align}
% \end{equation}



希腊字母这些如果不熟悉,可以去查找符号文件 \href{http://mirrors.ctan.org/info/symbols/comprehensive/symbols-a4.pdf}{symbols-a4.pdf} ,也可以去 \href{http://detexify.kirelabs.org/classify.html}{detexify} 网站手写识别。另外还有数学公式识别软件 \href{https://mathpix.com/}{mathpix} 。

\textbf{出于工作量,本模板未定义 definition、theorem 、lemma、corollary、assumption、conjecture、axiom、principle、problem、example、proof、solution 等环境,需要者可以自行添加。}

