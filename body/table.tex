\section{绘制普通三线表格}

表格应具有三线表格式,因此常用 booktabs宏包,其标准格式如\cref{tab:001}~所示。

\begin{table}[H]
    \caption{标准三线表格}\label{tab:001} \centering
    \vspace{-6pt}
    \begin{tabular}{ccccc}
        \toprule[1.5pt]
        $D$(in) & $P_u$(lbs) & $u_u$(in) & $\beta$ & $G_f$(psi.in)\\
        \midrule[1pt]
        5 & 269.8 & 0.000674 & 1.79 & 0.04089\\
        10 & 421.0 & 0.001035 & 3.59 & 0.04089\\
        20 & 640.2 & 0.001565 & 7.18 & 0.04089\\
        \bottomrule[1.5pt]
    \end{tabular}
\end{table}

其绘制表格的代码及其说明如下。
\begin{tcode}
    \begin{table}[!htbp]
        \caption[标签名]{中文标题}
        \begin{tabular}{cc...c}
            \toprule[1.5pt]
            表头第1个格   & 表头第2个格   & ... & 表头第n个格  \\
            \midrule[1pt]
            表中数据(1,1) & 表中数据(1,2) & ... & 表中数据(1,n)\\
            表中数据(2,1) & 表中数据(2,2) & ... & 表中数据(2,n)\\
            ...................................................\\
            表中数据(m,1) & 表中数据(m,2) & ... & 表中数据(m,n)\\
            \bottomrule[1.5pt]
        \end{tabular}
    \end{table}
\end{tcode}

\bigskip

table环境是一个将表格嵌入文本的浮动环境。tabular环境的必选参数由每列对应一个格式字符所组成:c表示居中,l表示左对齐,r表示右对齐,其总个数应与表的列数相同。此外,\verb|@{文本}|可以出现在任意两个上述的列格式之间,其中的文本将被插入每一行的同一位置。表格的各行以\verb|\\|分隔,同一行的各列则以\&分隔。 \verb|\toprule| 、\verb|\midrule| 和 \verb|\bottomrule| 三个命令是由booktabs宏包提供的,其中 \verb|\toprule| 和 \verb|\bottomrule| 分别用来绘制表格的第一条(表格最顶部)和第三条(表格最底部)水平线, \verb|\midrule| 用来绘制第二条(表头之下)水平线,且第一条和第三条水平线的线宽为 1.5pt ,第二条水平线的线宽为 1pt 。引用方法与图片的相同。

这里笔者给出常见用法,以供参考。

\begin{table}[H]
    \caption{符号说明}
    \vspace{-6pt}
    \centering
    \begin{tabularx}{0.9\textwidth}{c| >{\raggedright\arraybackslash}X}
        \toprule[1.5pt]
        \textbf{符号} & \textbf{内容说明} \\
        \midrule[1pt]
      
        $\epsilon_\mathrm{tol}$ & 算法误差容限 \\
        $\mathbf{a}_\odot^\mathrm{tidal}$ & 太阳引潮力项,按二阶梯度展开计算 \\
        $E$ & 偏近点角,通过牛顿迭代法求解开普勒方程 \\
        $\mathbf{\hat{P}},\mathbf{\hat{Q}}$ & 轨道面正交基向量,由$(\Omega,\omega,i)$计算 \\
        $\mathbf{k}_1,\mathbf{k}_2,...\mathbf{k}_{13}$ & 算法中间系数,对应13次函数估值 \\
        $B_{\mathrm{eff}}$ & 有效摄动阶数 \\
        \bottomrule[1.5pt]
    \end{tabularx}
\end{table}


\begin{table}[H]
\centering
\caption{2028 Olympic Medal Predictions (Top 14 Countries by Total Score)}
\vspace{-6pt} 
\label{tab:top14_medals}
\begin{tabular}{l|ccc|c} % 修正:5列(1左对齐 + 4居中对齐)
\toprule
\textbf{Country} & \textbf{Gold} & \textbf{Silver} & \textbf{Bronze} & \textbf{Total Score} \\
\midrule
United States         & 40.0 (38.0--46.3) & 40.4 (38.4--44.0) & 33.9 (32.2--42.0) & 112.0  \\
China                 & 36.7 (34.9--40.0) & 27.0 (25.6--30.4) & 24.0 (22.8--31.5) & 90.8    \\
Great Britain         & 17.1 (16.2--30.7) & 22.8 (21.7--28.5) & 26.6 (25.3--30.5) & 67.3     \\
Russia                & 14.3 (13.6--23.6) & 18.1 (17.2--20.9) & 20.9 (19.9--20.1) & 55.7    \\
France                & 12.1 (11.5--25.3) & 13.3 (12.6--26.0) & 21.2 (20.1--22.0) & 47.0    \\
Australia             & 12.1 (11.5--22.6) & 14.4 (13.7--19.4) & 16.2 (15.4--16.5) & 46.7    \\
Japan                 & 13.1 (12.4--20.5) & 14.9 (14.2--19.0) & 15.3 (14.5--17.0) & 45.0     \\
Italy                 & 12.1 (11.5--16.4) & 13.0 (12.3--13.6) & 17.4 (16.5--20.8) & 40.2     \\
Germany               & 12.1 (11.5--19.8) & 12.8 (12.2--18.7) & 13.4 (12.7--16.4) & 33.2     \\
Netherlands           & 7.1 (6.7--15.0)   & 8.0 (7.6--11.4)   & 12.8 (12.2--16.3) & 30.8     \\
South Korea           & 7.7 (7.3--13.0)   & 7.9 (7.5--9.5)    & 9.8 (9.3--10.9)   & 29.0    \\
New Zealand           & 4.5 (4.3--10.0)   & 5.9 (5.6--9.7)    & 5.8 (5.5--10.3)   & 20.1     \\
Brazil                & 4.8 (4.6--11.1)   & 6.1 (5.8--9.1)    & 7.4 (7.0--11.3)   & 20.0     \\
Canada                & 5.4 (5.1--10.2)   & 5.6 (5.3--7.3)    & 8.2 (7.8--11.0)   & 19.2    \\
\bottomrule
\end{tabular}
\smallskip
\vspace{2pt}
\footnotesize \textit{Note: Values are formatted as "median (confidence interval)". Countries are ranked by total score median.}
\end{table}


\begin{algorithm}[H]
\caption{预测误差与观测时间算法}
\label{alg:propagation}
\begin{algorithmic}[1]
\Require 
\Statex 初始轨道参数 $\mathbf{X}_0 \in \mathbb{R}^6$ (含误差协方差$\mathbf{P}_0$)
\Statex 摄动力模型 $\mathcal{M}$ (含J2至J6项、日月摄动、太阳光压等)
\Ensure
\Statex 位置误差统计量 $\mu_{\epsilon},\sigma_{\epsilon} \in \mathbb{R}^3$

\Procedure{误差传播分析}{$\mathbf{X}_0, \mathbf{P}_0, T_{max}$}
\State \textbf{步骤1:建立参考轨道}
\State 生成精密星历轨迹:
\quad $\mathbf{X}_{ref}(t) \gets \text{SPICE\_ephem}(t), t\in[0,T_{max}]$
\State 构造变分方程:
\quad $\frac{d}{dt}\Phi(t,0) = \mathbf{A}(t)\Phi(t,0)$

\State \textbf{步骤2:参数扰动采样}
\State 分解协方差矩阵:$\mathbf{P}_0 = \mathbf{L}\mathbf{L}^\top$ 
\For{$k=1$ \textbf{to} $N_{MC}$}
    \State 生成随机扰动:$\delta\mathbf{X}_k \gets \mathbf{L}\mathbf{z}_k,\ \mathbf{z}_k \sim \mathcal{N}(0,1)$
    \State 样本轨道参数:$\mathbf{X}_k^{(0)} \gets \mathbf{X}_0 + \delta\mathbf{X}_k$
\EndFor

\State \textbf{步骤3:轨道传播计算}
\For{$k=1$ \textbf{to} $N_{MC}$ \textbf{并行执行}}
    \State 数值积分轨道:
    \quad $\mathbf{X}_k(t) \gets \text{RK78}(\mathbf{X}_k^{(0)}, \mathcal{M}, T_{max})$
    \State 记录位置偏差:
    \quad $\Delta\mathbf{r}_k(t) \gets \mathbf{r}_k(t) - \mathbf{r}_{ref}(t)$
\EndFor

\State \textbf{步骤4:统计误差分布}
\State 计算时刻$t_j$的统计量:
\begin{align*}
\mu_{\epsilon}(t_j) &= \frac{1}{N_{MC}}\sum_{k=1}^{N_{MC}} \Delta\mathbf{r}_k(t_j) \\
\sigma_{\epsilon}^2(t_j) &= \frac{1}{N_{MC}-1}\sum_{k=1}^{N_{MC}} \|\Delta\mathbf{r}_k(t_j) - \mu_{\epsilon}\|^2
\end{align*}
\State \Return $\{\mu_{\epsilon}(t_j), \sigma_{\epsilon}(t_j)\}_{j=1}^M$
\EndProcedure
\end{algorithmic}
\end{algorithm}


\begin{table}[H]
\centering
\caption{Non-awarded Country}
\vspace{-6pt} 
\label{tab2}
\begin{tabular}{|c|c|c|c|c|c|c|c|}
\hline
BIZ & MAL & OMA & PLE & ESA & UNK & CAM & RWA \\ \hline
VNM & KIR & SOL & SWZ & MAW & HON & GEQ & BOL \\ \hline
STP & CGO & LBN & MDV & GAM & TUV & NRU & ANG \\ \hline
MHL & LES & YEM & SLE & VIN & LAO & CAF & MAD \\ \hline
MLI & SAA & BEN & ARU & GBS & AND & NBO & CRT \\ \hline
COK & FSM & LBR & GUI & SAM & VAN & MLT & LBA \\ \hline
SOM & NFL & ASA & SKN & LIE & BRU & COD & CAY \\ \hline
ROT & COM & NEP & SSD & CHA & MYA & GUM & TLS \\ \hline
BHU & BIH & IVB & YMD & PLW & YAR & BAN & SEY \\ \hline
MTN & ANT & PNG & NCA & RHO &   &   &   \\ \hline
\end{tabular}
\end{table}